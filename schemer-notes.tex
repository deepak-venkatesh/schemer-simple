% Created 2025-10-09 Thu 13:29
% Intended LaTeX compiler: pdflatex
\documentclass[11pt]{article}
\usepackage[utf8]{inputenc}
\usepackage[T1]{fontenc}
\usepackage{graphicx}
\usepackage{longtable}
\usepackage{wrapfig}
\usepackage{rotating}
\usepackage[normalem]{ulem}
\usepackage{amsmath}
\usepackage{amssymb}
\usepackage{capt-of}
\usepackage{hyperref}
\setlength{\parindent}{0pt}
\author{Deepak Venkatesh}
\date{\today}
\title{The Little Schemer Simplified v2}
\hypersetup{
 pdfauthor={Deepak Venkatesh},
 pdftitle={The Little Schemer Simplified v2},
 pdfkeywords={},
 pdfsubject={},
 pdfcreator={Emacs 29.4 (Org mode 9.6.15)}, 
 pdflang={English}}
\begin{document}

\maketitle
\tableofcontents

\newpage
\emph{Note:}

\vspace{1em}

These are my personal notes created to deepen my understanding of Lisp and programming in general. `The Little Schemer'
(4th edition) by Daniel Friedman and Matthias Felleisen is a remarkable book that teaches programming concepts in a
unique and playful way. It builds from first principles using only a small set of primitives, showing how powerful
ideas—such as recursion, functional programming, lambda functions, and interpreters—can be expressed using just
those few building blocks.

While the book uses Scheme, I prefer to work in Common Lisp and have adapted the examples accordingly. Despite its
lighthearted tone, the book is far from an easy read—it demands close attention and careful thought.

All mistakes in these notes, whether typographical or conceptual, are entirely my own. Any misinterpretations are as
well. I am still new to both Lisp and programming.

\newpage
\section{Foreword}
\label{sec:org3b82342}

By Gerald Sussman of MIT, co-author of the book SICP (the wizard book).

\vspace{1em}

Key Takeaway:
\emph{In order to be creative one must first gain control of the medium.}

\vspace{1em}

\begin{itemize}
\item Core skills are the first set of things required to master any pursuit.
\item Deep understanding is required to visualize beforehand the program which will be written.
\item Lisp provides freedom and flexibility (this is something which will only come in due course of time, as we keep
learning more about programming).
\item Lisp was initially conceived as a theoretical vehicle for recursion and symbolic algebra (this is the algebra we
have been taught in school \((a + b)^2 = a^2 + b^2 + 2ab\)).
\item In Lisp procedures are first class. Procedures are essentially a varaint of functions. A mathematical function maps
a given input to an output (domain - range/co-domain) but a procedure is a process to arrive at the result via
computation.
\item First Class basically means that procedure itself can be passed around as arguments to other procedures. Procedures
can be return values. They can also be stored in data structures. A similar corollary (though not exact) is
function of a function in pre-calculus.
\item Lisp programs can manipulate representations of Lisp programs - this likely refers to macros and how in Lisp code can
be treated as data.
\end{itemize}

\newpage
\section{Preface}
\label{sec:org5d21ec7}

Key Takeaway:
\emph{The goal of the book is to teach the reader to think recursively.}

\vspace{1em}

\begin{itemize}
\item Programs take data, apply a process on that data, and then produce some data.
\item Recursion is the act of defining an object or solving a problem in terms of itself.
\item 
\end{itemize}

\section{Toys}
\label{sec:org4a27f38}

\section{Do It, Do It Again, and Again, and Again \ldots{}}
\label{sec:org5cb434f}

\section{Cons the Magnificient}
\label{sec:org4005f26}

\section{Numbers Games}
\label{sec:org6d858f9}

\section{* Oh My Gawd *: It's Full of Stars}
\label{sec:org93923b1}

\section{Shadows}
\label{sec:org86e8f1b}

\section{Friends and Relations}
\label{sec:org9d6c68d}

\section{Lambda the Ultimate}
\label{sec:orgfd02769}

\section{\ldots{} and Again, and Again, and Again, \ldots{}}
\label{sec:org5116866}

\section{What Is the Value of All of This?}
\label{sec:org9268297}

\section{Intermission}
\label{sec:org7cc5137}

\section{The Ten Commandments}
\label{sec:org560fffe}

\section{The Five Rules}
\label{sec:orgcb4ac5f}
\end{document}
